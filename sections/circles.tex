\documentclass[class=article, crop=false]{standalone}
\usepackage[subpreambles=true]{standalone}
\usepackage{import}
%%\usepackage{booktabs}
%\usepackage{tikz}

%\usepackage[utf8]{inputenc}
\usepackage[subpreambles=true]{standalone}
\usepackage{import}
\usepackage{pgfplots}
\pgfplotsset{compat=newest}
\usepgfplotslibrary{groupplots}
\usepgfplotslibrary{dateplot}
\usepackage{caption}
%\usepackage{subcaption}
\usepackage{graphicx}
\usepackage{amsmath}
\usepackage{amssymb}
\usepackage[parfill]{parskip}
\usepackage{float}
\usepackage[bottom]{footmisc}

\setlength{\parindent}{2em}
\setlength{\parskip}{0.5em}
\usepackage{subcaption}
\usepackage{indentfirst}
\pgfplotsset{yticklabel style={text width=2em,align=right}}
\usepgfplotslibrary{external}
\usepackage{tikz}
\usepackage{shellesc}
\usetikzlibrary{external}
\tikzexternalize[shell escape=-enable-write18]
%\tikzexternalize[shell escape=-shell-escape]
\tikzset{external/system call={lualatex \tikzexternalcheckshellescape -halt-on-error -interaction=batchmode -jobname "\image" "\texsource"}}


% \usepackage{pgfplots}
% \usetikzlibrary{pgfplots.groupplots}
% \pgfplotsset{compat=1.9,height=0.3\textheight,legend cell align=left,tick scale binop=\times}
% \pgfplotsset{grid style={loosely dotted,color=darkgray!30!gray,line width=0.6pt},tick style={black,thin}}
% \pgfplotsset{every axis plot/.append style={line width=0.8pt}}
%
% \usepgfplotslibrary{external}
% % Für die Verwendung von 'external' müssen die folgenden Anpassungen in Abhängigkeit der
% % LaTeX Distribution durchgeführt werden:
%
% % fuer Texlive: pdflatex.exe -shell-escape -synctex=1 -interaction=nonstopmode %.tex
% \tikzexternalize[shell escape=-shell-escape]   % fuer TeXLive
%
% % fuer MikTeX:  pdflatex.exe -enable-write18 -synctex=1 -interaction=nonstopmode %.tex
% %\tikzexternalize[shell escape=-enable-write18] % fuer MikTex
%
%
%
\tikzsetexternalprefix{graphics/} % Ordner muss ev. zuerst haendisch erstellt werden

\begin{document}

\section{Localization drift}\label{sec:circles}
For our next experiment we decided to test our robot's potential drift over time while running robot\_localization's ekf\_localization\_node.

We ran the robot in an octagon pattern with a side length of approximately 66cm-s. We chose this shape because of relative ease to drive the robot straight then turn and its similarity to a circle. By driving the robot in a loop through this octagonal path, we can observe the drift in its localization.

As an input we used our fake wheel encoder as an approximate velocity input $\dot{x}$ in the robots relative x direction, the accelerometer's relative x direction as $\ddot{x}$ and the gyroscope's angular z velocity as a $\dot{\psi}$. We estimated the standard deviation of the fake wheel encoder in relation to the accelerometer's to $\frac{1}{9}$.

\vspace{0.5cm}

\begin{figure}[H]
\begin{flushleft}
  \begin{tikzpicture}
    \begin{axis}[
      unit vector ratio =1 1,
      yticklabel style={
        /pgf/number format/fixed,
        /pgf/number format/precision=5
      },
      width=12cm,
      height=12cm,
      xtick distance = 0.2,
      ytick distance = 0.2,
      xlabel={x $[m]$},
      ylabel={y $[m]$},
      grid=both,
      grid style={
        line width=.1pt,
        draw=gray!10},
        major grid style={
          line width=.2pt,
          draw=gray!50
        },
      ]
      \addplot+[color=blue, mark=none,line width=1pt,mark size=1pt] table {plots/loop_xy.csv};
    \end{axis}
  \end{tikzpicture}
\end{flushleft}
\caption{Octagon x-y plot}\label{fig:octagon_xy}
\end{figure}

As we can see in Figure \ref{fig:octagon_xy}, the octagon shape of the path drifts in the x direction about 1.25m. This is due to the gyroscope drifting over the course of the experiment. This drift cannot be observed in y or $\psi$ as seen in Figure \ref{fig:octagon_xyyaw}, because the discrepency occures only in reality and can be detected only by another sensor(most commonly a magnetometer).

\vspace{0.5cm}

\begin{figure}[H]
\begin{flushleft}
  \begin{tikzpicture}
    \begin{axis}[
      yticklabel style={
        /pgf/number format/fixed,
        /pgf/number format/precision=5
      },
      width=12cm,
      height=5cm,
      xtick distance = 50,
      ytick distance = 0.5,
      xlabel={Time $[s]$},
      ylabel={x $[m] $},
      grid=both,
      grid style={
        line width=.1pt,
        draw=gray!10},
        major grid style={
          line width=.2pt,
          draw=gray!50
        },
      ]
      \addplot+[color=blue, mark=none,line width=1pt,mark size=1pt] table {plots/loop_x.csv};
    \end{axis}
  \end{tikzpicture}

  \begin{tikzpicture}
    \begin{axis}[
      yticklabel style={
        /pgf/number format/fixed,
        /pgf/number format/precision=5
      },
      width=12cm,
      height=5cm,
      xtick distance = 50,
      ytick distance = 0.5,
      xlabel={Time $[s]$},
      ylabel={y $[m] $},
      grid=both,
      grid style={
        line width=.1pt,
        draw=gray!10},
        major grid style={
          line width=.2pt,
          draw=gray!50
        },
      ]
      \addplot+[color=blue, mark=none,line width=1pt,mark size=1pt] table {plots/loop_y.csv};
    \end{axis}
  \end{tikzpicture}
  \begin{tikzpicture}
    \begin{axis}[
      yticklabel style={
        /pgf/number format/fixed,
        /pgf/number format/precision=5
      },
      width=12cm,
      height=5cm,
      xtick distance = 50,
      ytick distance = 1.5,
      xlabel={Time $[s]$},
      ylabel={$\psi$ $[rad] $},
      grid=both,
      grid style={
        line width=.1pt,
        draw=gray!10},
        major grid style={
          line width=.2pt,
          draw=gray!50
        },
      ]
      \addplot+[color=blue, mark=none,line width=1pt,mark size=1pt] table {plots/loop_yaw.csv};
    \end{axis}
  \end{tikzpicture}
\end{flushleft}
\caption{Octagon $x$, $y$ and $\psi$ plots}\label{fig:octagon_xyyaw}
\end{figure}

\end{document}
