\documentclass[class=report, crop=false]{standalone}
\usepackage[subpreambles=true]{standalone}
\usepackage{import}
\begin{document}

\section{Software}\label{sec:software}
In this section we present the software used on the platform and which alternatives were considered.

%%%%%%%%%%%%%%%%%%%%%%%%%%%%%%%%%%%%%%%%%%%%%%%%%%%%%%%%%%%%%%%%%%%%%%%%%%%%%%%%%%%%%
\subsection{Operating system}\label{subsec:os}
When deciding on what OS to run on the Raspberry Pi 3, we considered the classic Ubuntu 16.04, Ubuntu Core and Raspbian.

Raspbian is not officially supported by ROS Kinetic and compiling directly on the Raspberry Pi is not ideal due to the limited 1-GB RAM and CPU speed. Ubuntu Core was ruled out due to it being in early development and ease of use.

We chose the classic Ubuntu 16.04 developed for the Raspberry Pi 2 from Canonical Ltd. After following the instructions found on the official page\footnotemark the image was adjusted to be used with our Raspberry Pi 3B.

\footnotetext{https://wiki.ubuntu.com/ARM/RaspberryPi}
In order to communicate with our platform remotely, we set up a wireless access point on the Raspberry Pi and use secure shell(ssh) for communication.

On the Arduino Uno we used a custom firmware\footnotemark that behaves like a node for the Raspberry Pi using the rosserial\_arduino ROS package. This way we can directly control the actuators and used the sensors connected to the Arduino with ROS messages.

\footnotetext{https://github.com/braineniac/garry-firmware}

%%%%%%%%%%%%%%%%%%%%%%%%%%%%%%%%%%%%%%%%%%%%%%%%%%%%%%%%%%%%%%%%%%%%%%%%%%%%%%%%%%%%%
\subsection{ROS}\label{subsec:ros}
The Robot Operating System(ROS) is a free and open-source middleware that uses nodes and message parsing to implement a highly configurable software stack intended for a single robot.

Support for ROS enables the use of the wealth of software developed by the ROS community and easy adaptability of the framework for any use case envisioned.

On our platform we used ROS Kinetic, because at the time of writing it has the best support for Raspberry Pi packages.
%%%%%%%%%%%%%%%%%%%%%%%%%%%%%%%%%%%%%%%%%%%%%%%%%%%%%%%%%%%%%%%%%%%%%%%%%%%%%%%%%%%%%

\subsection{Drivers}\label{subsec:drivers}
The Razor IMU is connected directly to the Rasberry Pi and we used the razor\_imu\_9dof\footnotemark  ROS package for reading its data. We flashed the included default firmware onto the Razor board. Further firmware tweaking for higher sensor rate doesn't yield considerable improvements and is not recommended by the software author.

\footnotetext{http://wiki.ros.org/razor\_imu\_9dof}

In a previous attempt, we developed a ROS driver\footnote{https://github.com/braineniac/rpi\_mpu6050} for the MPU6050 that was connected via $ I^2C $ to the Raspberry Pi, but the underlying libraries proved to be unreliable and was completely replaced by a more stable and reliable solution, the Razor IMU.

The servos are connected to the Arduino Uno's PWM pins and can be controlled by by ROS topics /servo\_upper and /servo\_lower directly. The motors are accessible through the customary /cmd\_vel topic and are connected to the Arduino Motor Shield. A differential drive was not implemented, depending on the direction a motor turns on or off and runs forwards or backwards for about 100ms, then it stops for security reasons.

As for the cameras we used the the usb\_cam\footnote{http://wiki.ros.org/usb\_cam} ROS package for the Logitech Webcam and the raspicam\_node\footnote{https://github.com/UbiquityRobotics/raspicam\_node} from Ubiquity Robotics to provide a ROS interface. Both cameras were calibrated with a 7x9 checkerboard.

The joystick was handled by the joy\footnotetext{http://wiki.ros.org/joy} ROS package and the input is repeated every 100ms for smooth operation of the motors by our teleop\_node\footnotetext{https://github.com/braineniac/garry-robot}.

%%%%%%%%%%%%%%%%%%%%%%%%%%%%%%%%%%%%%%%%%%%%%%%%%%%%%%%%%%%%%%%%%%%%%%%%%%%%%%%%%%%%%

%\import{subsections/}{kalman}

%%%%%%%%%%%%%%%%%%%%%%%%%%%%%%%%%%%%%%%%%%%%%%%%%%%%%%%%%%%%%%%%%%%%%%%%%%%%%%%%%%%%%
\subsection{ORB-SLAM2}\label{subsec:orbslam2}

In order to evaluate the platform a complete simultaneous localization and mapping solution(SLAM) described in the papers\cite{mur2017orb}\cite{mur2015orb} by Mur-Artal, Ra\'ul and Tard{\'o}s, Juan D was attempted. We used a Monocular Camera input from our Logitech C905 Webcam mounted on the front of the platform.

Since publication, the source code\footnote{https://github.com/raulmur/ORB\_SLAM2} was not maintained by the authors so we created our own fork\footnotemark in order to immediately evaluate and plot the trajectory with the python package evo\cite{grupp2017evo}.

\footnotetext{https://github.com/braineniac/ORB\_SLAM2}

In our tests the CPU of the Raspberry Pi proved not to be powerful enough to run ORB-SLAM-2 in real time while the robot was moving. The very slow framerate of around 1-2 fps in Pangolin\footnotemark wasn't nearly sufficient for even automatic initialization described in the authors previous paper\cite{mur2015orb} and was due to the CPU being overloaded. The evaluation of the 2017 paper\cite{mur2017orb} was done on an Intel Core i7-4790 and 16-GB RAM, which is has considerably more processing power.

\footnotetext{https://github.com/stevenlovegrove/Pangolin}

Our conclusion is that using the published ORB-SLAM2 code in its current state is not possible in real time on the Raspberry Pi 3B and would require GPU acceleration or other code optimizations.

\end{document}
