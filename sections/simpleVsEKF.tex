\documentclass[class=article, crop=false]{standalone}
%\usepackage[subpreambles=true]{standalone}
\usepackage{import}
%%\usepackage{booktabs}
%\usepackage{tikz}

%\usepackage[utf8]{inputenc}
\usepackage[subpreambles=true]{standalone}
\usepackage{import}
\usepackage{pgfplots}
\pgfplotsset{compat=newest}
\usepgfplotslibrary{groupplots}
\usepgfplotslibrary{dateplot}
\usepackage{caption}
\usepackage{subcaption}
\usepackage{graphicx}
\usepackage{amsmath}
\usepackage{amssymb}
\usepackage[parfill]{parskip}
\usepackage{float}

% \usepackage{pgfplots}
% \usetikzlibrary{pgfplots.groupplots}
% \pgfplotsset{compat=1.9,height=0.3\textheight,legend cell align=left,tick scale binop=\times}
% \pgfplotsset{grid style={loosely dotted,color=darkgray!30!gray,line width=0.6pt},tick style={black,thin}}
% \pgfplotsset{every axis plot/.append style={line width=0.8pt}}
%
% \usepgfplotslibrary{external}
% % Für die Verwendung von 'external' müssen die folgenden Anpassungen in Abhängigkeit der
% % LaTeX Distribution durchgeführt werden:
%
% % fuer Texlive: pdflatex.exe -shell-escape -synctex=1 -interaction=nonstopmode %.tex
% \tikzexternalize[shell escape=-shell-escape]   % fuer TeXLive
%
% % fuer MikTeX:  pdflatex.exe -enable-write18 -synctex=1 -interaction=nonstopmode %.tex
% %\tikzexternalize[shell escape=-enable-write18] % fuer MikTex
%
%
%
% \tikzsetexternalprefix{graphics/pgfplots/} % Ordner muss ev. zuerst haendisch erstellt werden

\begin{document}
\pgfplotsset{width=14cm,compat=1.9}
\section{Simple Kalman Filter comparison}\label{sec:expcomp}

For our first experiment we compare our Simple Kalman Filter(SKF) to the robot\_localization packgage's Extended Kalman Filter(EKF).

As the system input we used the Gamepads button press as an approximate velocity input, which we previously measured, and the acceleration from our Razor IMU. This can be seen in Figure \ref{fig:input}.
\begin{figure}
\begin{flushleft}
  \begin{tikzpicture}
    \begin{axis}[
      yticklabel style={
        /pgf/number format/fixed,
        /pgf/number format/precision=5
      },
      width=12cm,
      height=5cm,
      xtick distance=0.5,
      xlabel={$ Time [s] $},
      ylabel={$ \textbf{u}^0 [m/s] $}
      ]
      \addplot+[color=blue, mark=none,line width=1pt,mark size=1pt] table {../plots/real_input_vel.csv};
    \end{axis}
  \end{tikzpicture}
\end{flushleft}

\begin{flushleft}
  \begin{tikzpicture}
    \begin{axis}[
      yticklabel style={
        /pgf/number format/fixed,
        /pgf/number format/precision=5
      },
      width=12cm,
      height=5cm,
      xtick distance=0.5,
      xlabel={$ Time [s] $},
      ylabel={$ \textbf{u}^1 [m/s^2] $}
      ]
      \addplot+[color=blue, mark=none,line width=1pt,mark size=1pt] table {../plots/real_input_accel.csv};
    \end{axis}
  \end{tikzpicture}
\end{flushleft}
\caption{$ \textbf{u} $ input to the system }\label{fig:input}
\end{figure}

The robot was moved in a straight line and the messages from the Gamepad and IMU were recorded with rosbag. Launching the filter nodes was done offline with the rosbag recording for proper direct comparison of the estimated states. In this case the adaptive ability of the Simple Kalman Filter was disabled. We used a ratio of $ \textbf{r}_{k} = \frac{1}{3} $.

\begin{figure}
\begin{flushleft}
\begin{tikzpicture}
  \begin{axis}[
    yticklabel style={
      /pgf/number format/fixed,
      /pgf/number format/precision=5
    },
    width=12cm,
    height=5cm,
    xtick distance=0.5,
    xlabel={$Time [s]$},
    ylabel={$x state [m]$},
    legend pos=south east
    ]
    \addplot+[smooth,color=blue, mark=none,line width=1pt,mark size=1pt] table {../plots/ekf_pos.csv};
    \addlegendentry{EKF}
    \addplot+[smooth,color=red, mark=none,line width=1pt,mark size=1pt, dashed] table {../plots/real_robot_dist_None.csv};
    \addlegendentry{SKF}
  \end{axis}
\end{tikzpicture}
\end{flushleft}

\begin{flushleft}
\begin{tikzpicture}
  \begin{axis}[
    yticklabel style={
      /pgf/number format/fixed,
      /pgf/number format/precision=5
    },
    width=12cm,
    height=5cm,
    xtick distance=0.5,
    xlabel={$Time [s]$},
    ylabel={$\dot{x} state [m/s]$},
    legend pos=north east
    ]
    \addplot+[color=blue, mark=none,line width=1pt,mark size=1pt, dashed] table {../plots/ekf_vel.csv};
    \addlegendentry{EKF}
    \addplot+[color=red, mark=none,line width=1pt,mark size=1pt] table {../plots/real_robot_vel_None.csv};
    \addlegendentry{SKF}
  \end{axis}
\end{tikzpicture}\label{fig:xdotcompare}
\end{flushleft}
\caption{$ x $ and $\dot{x} $ state estimate comparison}\label{fig:statescompare}
\end{figure}

We can observe the difference in Figure \ref{fig:statescompare} in how the acceleration during movement is seen in the $\dot{x} $ state estimation. EFK deals better with such erratic changes in $ \textbf{u}^1 $ and smooths them out while our Simple Kalman Filter remains relatively noisy in that section. EKF also reacts sooner than SKF to the jump in $ \textbf{u}^0 $.

Further optimisations with adaptive covariances will be explored in the next experiment.

\end{document}
