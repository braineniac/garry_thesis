\documentclass[class=report, crop=false]{standalone}
\usepackage[subpreambles=true]{standalone}
\usepackage{import}
%%\usepackage{booktabs}
%\usepackage{tikz}

%\usepackage[utf8]{inputenc}
\usepackage[subpreambles=true]{standalone}
\usepackage{import}
\usepackage{pgfplots}
\pgfplotsset{compat=newest}
\usepgfplotslibrary{groupplots}
\usepgfplotslibrary{dateplot}
\usepackage{caption}
\usepackage{subcaption}
\usepackage{graphicx}
\usepackage{amsmath}
\usepackage{amssymb}
\usepackage[parfill]{parskip}
\usepackage{float}

% \usepackage{pgfplots}
% \usetikzlibrary{pgfplots.groupplots}
% \pgfplotsset{compat=1.9,height=0.3\textheight,legend cell align=left,tick scale binop=\times}
% \pgfplotsset{grid style={loosely dotted,color=darkgray!30!gray,line width=0.6pt},tick style={black,thin}}
% \pgfplotsset{every axis plot/.append style={line width=0.8pt}}
%
% \usepgfplotslibrary{external}
% % Für die Verwendung von 'external' müssen die folgenden Anpassungen in Abhängigkeit der
% % LaTeX Distribution durchgeführt werden:
%
% % fuer Texlive: pdflatex.exe -shell-escape -synctex=1 -interaction=nonstopmode %.tex
% \tikzexternalize[shell escape=-shell-escape]   % fuer TeXLive
%
% % fuer MikTeX:  pdflatex.exe -enable-write18 -synctex=1 -interaction=nonstopmode %.tex
% %\tikzexternalize[shell escape=-enable-write18] % fuer MikTex
%
%
%
% \tikzsetexternalprefix{graphics/pgfplots/} % Ordner muss ev. zuerst haendisch erstellt werden


\begin{document}

\section{Straight line}\label{sec:straightline}

Now that we have all of our parameters set, we will try to recreate the simulation done in section \ref{subsubsec:ekf} with the $\mu_v$ tuning experiment and see if our EKF improves the x state estimation and if our rotation parameters influences it or not.

\begin{figure}[H]
 \begin{flushleft}
  \begin{tikzpicture}
   \begin{axis}[
     yticklabel style={
       /pgf/number format/fixed,
       /pgf/number format/precision=5
      },
     width=12cm,
     height=5cm,
     xmax=16,
     xmin=-1,
     xtick distance=1,
     ytick distance=1,
     xlabel={Time $[s]$},
     ylabel={x state [m]},
     legend pos=north west,
     grid=both,
     grid style={
       line width=.1pt,
       draw=gray!10},
     major grid style={
       line width=.2pt,
       draw=gray!50
      },
     domain=0:36,
    ]
    \addplot+[color=red, dashed,mark=none,line width=1pt,mark size=1pt] table {plots/straight_line0_x0.csv};
    \addlegendentry{KF}
    \addplot+[color=blue, mark=none,line width=1pt,mark size=1pt] table {plots/straight_line1_x0.csv};
    \addlegendentry{EKF}
   \end{axis}
  \end{tikzpicture}
  \begin{tikzpicture}
   \begin{axis}[
     yticklabel style={
       /pgf/number format/fixed,
       /pgf/number format/precision=5
      },
     width=12cm,
     height=5cm,
     xmin=-1,
     xmax=16,
     ytick distance = 0.2,
     xtick distance=1,
     xlabel={Time $[s]$},
     ylabel={v state [m/s]},
     legend pos=north west,
     grid=both,
     grid style={
       line width=.1pt,
       draw=gray!10},
     major grid style={
       line width=.2pt,
       draw=gray!50
      },
     domain=0:36,
    ]
    \addplot+[color=red, dashed,mark=none,line width=1pt,mark size=1pt] table {plots/straight_line0_x2.csv};
    \addlegendentry{KF}
    \addplot+[color=blue, mark=none,line width=1pt,mark size=1pt] table {plots/straight_line1_x2.csv};
    \addlegendentry{EKF}
   \end{axis}
  \end{tikzpicture}
 \end{flushleft}
 \caption{Straight line experiment}\label{fig:straightline}
\end{figure}

Our x state estimation barely changes, since the additional corrections done by the acceleromater during the input jumps have insignificant impact on the x state estimation.

Since the distance is still 5 meters, the influence of the rotational parameters is minimal.

\end{document}
