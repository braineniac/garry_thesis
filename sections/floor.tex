\documentclass[class=report, crop=false]{standalone}
\usepackage[subpreambles=true]{standalone}
\usepackage{import}
%%\usepackage{booktabs}
%\usepackage{tikz}

%\usepackage[utf8]{inputenc}
\usepackage[subpreambles=true]{standalone}
\usepackage{import}
\usepackage{pgfplots}
\pgfplotsset{compat=newest}
\usepgfplotslibrary{groupplots}
\usepgfplotslibrary{dateplot}
\usepackage{caption}
\usepackage{subcaption}
\usepackage{graphicx}
\usepackage{amsmath}
\usepackage{amssymb}
\usepackage[parfill]{parskip}
\usepackage{float}

% \usepackage{pgfplots}
% \usetikzlibrary{pgfplots.groupplots}
% \pgfplotsset{compat=1.9,height=0.3\textheight,legend cell align=left,tick scale binop=\times}
% \pgfplotsset{grid style={loosely dotted,color=darkgray!30!gray,line width=0.6pt},tick style={black,thin}}
% \pgfplotsset{every axis plot/.append style={line width=0.8pt}}
%
% \usepgfplotslibrary{external}
% % Für die Verwendung von 'external' müssen die folgenden Anpassungen in Abhängigkeit der
% % LaTeX Distribution durchgeführt werden:
%
% % fuer Texlive: pdflatex.exe -shell-escape -synctex=1 -interaction=nonstopmode %.tex
% \tikzexternalize[shell escape=-shell-escape]   % fuer TeXLive
%
% % fuer MikTeX:  pdflatex.exe -enable-write18 -synctex=1 -interaction=nonstopmode %.tex
% %\tikzexternalize[shell escape=-enable-write18] % fuer MikTex
%
%
%
% \tikzsetexternalprefix{graphics/pgfplots/} % Ordner muss ev. zuerst haendisch erstellt werden

\begin{document}

\section{Corridor mapping}\label{sec:floor}

In this experiment we do a more realistic test to see differences in our Kalman filter with and without adaptive process noise.

We drive around in a corridor about 33 meters in length and 2.5 meters wide. We also drove through a small side corridor on the left which measured 10 meters long and 2.5 meters wide. There were also some glass doors that we maneuvered around.

\begin{center}
\begin{figure}[H]
 \begin{flushleft}
  \begin{tikzpicture}
   \begin{axis}[
     yticklabel style={
       /pgf/number format/fixed,
       /pgf/number format/precision=5
      },
     width=12cm,
     height=12cm,
     xtick distance = 3,
     ytick distance = 3,
     xlabel={x state [m]},
     ylabel={y state [m]},
     legend pos=north west,
     grid=both,
     grid style={
       line width=.1pt,
       draw=gray!10},
     major grid style={
       line width=.2pt,
       draw=gray!50
      },
    ]
    \addplot+[color=red, mark=none,line width=1pt,mark size=1pt, dashed] table {plots/floor0_x0x1.csv};
    \addlegendentry{KF}
    \addplot+[color=blue, mark=none,line width=1pt,mark size=1pt] table {plots/floor1_x0x1.csv};
    \addlegendentry{KF(APN)}
   \end{axis}
  \end{tikzpicture}
 \end{flushleft}
 \caption{Comparison of the xy-trajectory estimates with the Kalman filter and Kalman filter with adaptive process noise after driving the robot close to the walls of the corridor.}\label{fig:floorxy}
\end{figure}
\end{center}

Figure \ref{fig:floorxy} shows that our filters can't estimate small turns accurately, but we can still see the floor shape. The error seems to be consistent since when we return to our starting point, its in a distance of less than 1 meter.

The Kalman filter with adaptive process noise is further away from the starting point, because it relies more on the gyroscope during turns. Balancing model and gyroscope achieves better results than relying just on the gyroscope. This can be seen in Figure \ref{fig:floormistuned}, where we increased the ratio $\varrho_{\eta\chi}$ by 150\%.

\begin{center}
\begin{figure}[H]
 \begin{flushleft}
  \begin{tikzpicture}
   \begin{axis}[
     yticklabel style={
       /pgf/number format/fixed,
       /pgf/number format/precision=5
      },
     width=12cm,
     height=12cm,
     xtick distance = 3,
     ytick distance = 3,
     xlabel={x state [m]},
     ylabel={y state [m]},
     legend pos=north west,
     grid=both,
     grid style={
       line width=.1pt,
       draw=gray!10},
     major grid style={
       line width=.2pt,
       draw=gray!50
      },
    ]
    \addplot+[color=red, mark=none,line width=1pt,mark size=1pt, dashed] table {plots/floor0_x0x1.csv};
    \addlegendentry{KF}
    \addplot+[color=blue, mark=none,line width=1pt,mark size=1pt] table {plots/floor2_x0x1.csv};
    \addlegendentry{mistuned KF}
   \end{axis}
  \end{tikzpicture}
 \end{flushleft}
 \caption{Effects of mistuning the $\varrho_{\eta\chi}$ parameter by 150\% on the x-y trajectory after driving the robot close to the walls of the corridor.}\label{fig:floormistuned}
\end{figure}
\end{center}

\end{document}
