\addchap*{Abstract}

The rise in popularity of the maker culture, which democratises access to STEM and other tech-rich domains, has enabled widespread production of cheap components which can be utilised for research purposes.\\
This project aims to use off the shelf components to build a rudimentary robot with basic odometrical capabilities while keeping the budget minimal.\\
Through exploring the limits of such a construct, we can speculate what is reasonably possible to build with today's maker products that can
be used as a base for an extensible platform to conduct scientific and engineering research while keeping its cost to a bare minimum.
