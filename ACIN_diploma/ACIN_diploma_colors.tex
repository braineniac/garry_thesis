% Definition eines einheitlichen Farbschemas und Definition von styles.
%=======================================================================

% Um ein einheitliches Farbschema zu gewährleisten werden hier einige Farben und
% davon abgeleitete Variationen definiert. Diese Farben werden in der Vorlage
% bereits für die entsprechenden Befehle (Links, Hervorhebung, usw.) verwendet.
% Die Farben sollten auch in Grafiken verwendet werden. Die RGB- bzw. CMYK Codes
% werden hierzu in der Dokumentation angegeben.
% Es werden außerdem einige styles hier definiert, die von verschiedenen
% Befehlen verwendet werden.


% Definition der Grundfarben
%============================

% \providecolor{acin_red}{rgb/cmyk}{0.7294,0.0706,0.1686/0.03,1,0.66,0.12}
% \providecolor{test}{rgb}{0.85,0.0,0.22}
% \providecolor{test2}{rgb}{0,0.47,0.85}
% \providecolor{acin_gray}{cmyk}{0.2,0.15,0.11,0.4}
% \providecolor{acin_yellow}{cmyk}{0.01000,0.20000,0.72000,0.00000}
% \providecolor{TU_blue}{cmyk}{1,0.38,0,0.15}
% \definecolor{acin_green}{rgb}{0,0.75,0.25}

\definecolor{acin_red}{RGB}{186, 18, 43}
\definecolor{acin_gray}{RGB}{176, 176, 176}
\definecolor{acin_yellow}{RGB}{252, 204, 71}
\definecolor{acin_green}{RGB}{0, 190, 65}
\definecolor{TU_blue}{RGB}{0, 102, 153}
\definecolor{test2}{rgb}{0,0.47,0.85}
\definecolor{TU_gray}{RGB}{102, 102, 102}
\providecolor{test}{rgb}{0.85, 0, 0.22}

\definecolor{acin_red_complement}{RGB}{50,255,0}
\definecolor{TU_blue_complement}{RGB}{255,106,0}
\definecolor{acin_yellow_variant}{RGB}{255,238,0}
\definecolor{acin_green_variant}{RGB}{25,150,0}
\definecolor{acin_blue_variant}{RGB}{19,93,255}
\definecolor{TU_green_variant}{RGB}{157,217,0}
\definecolor{TU_pink_variant}{RGB}{202,0,90}
\definecolor{TU_violet_variant}{RGB}{218,33,255}

% Definition von abgeleiteten Farben
%====================================

\colorlet{acin_red_lightvar}{acin_red!80}
\colorlet{acin_red_verylightvar}{acin_red!25!white!97!acin_gray}

% Alternative Namen
%===================
\definecolor{red}{named}{acin_red}
\definecolor{gray}{named}{TU_gray}
\definecolor{yellow}{named}{acin_yellow}
\definecolor{green}{named}{acin_green}
\definecolor{blue}{named}{TU_blue}

\definecolor{light_red}{named}{acin_red_lightvar}


% Festlegen der Farben von Gestaltungselementen
%===============================================
% \ifthenelse{\boolean{ACINcolored}}{
	% farbige Version des Skripts
	\definecolor{emph}{named}{acin_red} % Farbe für Hervorhebungen
	\definecolor{env}{named}{TU_blue} % Farbe für Umgebungen
	\definecolor{env_bck}{RGB}{235,235,235} % Hintergrundfarbe für Umgebungen
	\definecolor{link_color}{named}{TU_blue} % Links
	\definecolor{url_color}{named}{TU_blue} % urls
	\definecolor{cite_color}{named}{acin_green} % Zitate
	
	\definecolor{HinweisGreen}{named}{acin_green_variant}
	\colorlet{HinweisGreen_bck}{HinweisGreen!15!white}
	\definecolor{HinweisYellow}{RGB}{250, 170,	40}
	\colorlet{HinweisYellow_bck}{HinweisYellow!10!white}
	\definecolor{HinweisBlue}{named}{TU_blue}
	\colorlet{HinweisBlue_bck}{HinweisBlue!25!white}
	\colorlet{HinweisBW}{black!60!white}
	\colorlet{HinweisBW_bck}{HinweisBW!15!white}
	
	\definecolor{EditorNote}{named}{orange}
	\colorlet{EditorNote_bck}{EditorNote!25!white}
	
	\colorlet{eqn}{TU_blue!70!black!15!white}
	\colorlet{eqn_bck}{TU_blue!70!black!15!white}
	
	\colorlet{proof}{TU_blue!70!black!15!white}
	
	% Farben zur Hervorhebung in Formeln
	\definecolor{EqnEmph1}{named}{acin_yellow}
	\colorlet{EqnEmph2}{TU_blue!70!black!15!white}
	\definecolor{EqnEmph3}{named}{acin_red_verylightvar}
% }{
% 	% schwarz-weiß Version des Skripts
% 	\definecolor{emph}{named}{black} % Farbe für Hervorhebungen
% 	\definecolor{env}{named}{black} % Farbe für Umgebungen
% 	\definecolor{env_bck}{RGB}{235,235,235} % Hintergrundfarbe für Umgebungen
% 	\definecolor{link_color}{named}{black} % Links
% 	\definecolor{url_color}{named}{black} % urls
% 	\definecolor{cite_color}{named}{black} % Zitate
% 	
% 	\definecolor{EditorNote}{RGB}{40,40,40}
% 	\definecolor{EditorNote_bck}{RGB}{175,175,175}
% 	\definecolor{HinweisGreen}{RGB}{40,40,40}
% 	\definecolor{HinweisGreen_bck}{RGB}{235,235,235}
% 	\definecolor{HinweisYellow}{RGB}{40,40,40}
% 	\definecolor{HinweisYellow_bck}{RGB}{235,235,235}
% 	
% 	\definecolor{eqn}{RGB}{215,215,215}
% 	\definecolor{eqn_bck}{RGB}{215,215,215}
% 	
% 	\definecolor{proof}{RGB}{215,215,215}
% 	
% 	% Farben zur Hervorhebung in Formeln
% 	\definecolor{EqnEmph1}{gray}{0.85}
% 	\definecolor{EqnEmph2}{gray}{0.90}
% 	\definecolor{EqnEmph3}{gray}{0.95}
% }



