\documentclass[class=report, crop=false]{standalone}
\usepackage[subpreambles=true]{standalone}
%%\usepackage{booktabs}
%\usepackage{tikz}

%\usepackage[utf8]{inputenc}
\usepackage[subpreambles=true]{standalone}
\usepackage{import}
\usepackage{pgfplots}
\pgfplotsset{compat=newest}
\usepgfplotslibrary{groupplots}
\usepgfplotslibrary{dateplot}
\usepackage{caption}
\usepackage{subcaption}
\usepackage{graphicx}
\usepackage{amsmath}
\usepackage{amssymb}
\usepackage[parfill]{parskip}
\usepackage{float}

% \usepackage{pgfplots}
% \usetikzlibrary{pgfplots.groupplots}
% \pgfplotsset{compat=1.9,height=0.3\textheight,legend cell align=left,tick scale binop=\times}
% \pgfplotsset{grid style={loosely dotted,color=darkgray!30!gray,line width=0.6pt},tick style={black,thin}}
% \pgfplotsset{every axis plot/.append style={line width=0.8pt}}
%
% \usepgfplotslibrary{external}
% % Für die Verwendung von 'external' müssen die folgenden Anpassungen in Abhängigkeit der
% % LaTeX Distribution durchgeführt werden:
%
% % fuer Texlive: pdflatex.exe -shell-escape -synctex=1 -interaction=nonstopmode %.tex
% \tikzexternalize[shell escape=-shell-escape]   % fuer TeXLive
%
% % fuer MikTeX:  pdflatex.exe -enable-write18 -synctex=1 -interaction=nonstopmode %.tex
% %\tikzexternalize[shell escape=-enable-write18] % fuer MikTex
%
%
%
% \tikzsetexternalprefix{graphics/pgfplots/} % Ordner muss ev. zuerst haendisch erstellt werden



\begin{document}

\chapter{Conclusion}\label{cha:conclusion}

In the thesis we introduced a platform built out of remains of a toy robot and equipped it with components that make it a cost effective solution for research.

We have implemented a system using custom Kalman filters for odometry using joystick signals as a control input and IMU sensors as an output.

Due to the accelerometer's lower than 1 signal-to-noise ratio(SNR) we supressed its impact on the state estimation with a higher process noise covariance $\textbf{Q}_k$. The noise originated from the DC motor's vibrations.

From the conducted experiments we have learned that small errors in the rotation accumulate easily and cause deformation of the overall trajectory.

Improving the odometry with visual SLAM like ORB-SLAM2\cite{mur2017orb} is not computationally feasable on the Raspberry Pi 3 without further code optimizations.

\end{document}
