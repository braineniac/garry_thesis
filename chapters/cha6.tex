\documentclass[class=article, crop=false]{standalone}
\usepackage[subpreambles=true]{standalone}
\usepackage{import}

\begin{document}

\chapter{Outlook}\label{cha:outlook}

The gyroscope's drift seen in Section \ref{sec:circles} could be accounted for in a loop closing method if the path is relatively short or continuously correct it with a magnetometer.

The underlying electrical structure could be improved for reliability and ease of use similar to the Turtlebot 3 Burger described in Section \ref{sec:turtlebot3burger}. Either a custom PCB or an Arduino Shield designed for the platform would make it easier to upgrade and maintain it.

The platform could be also equipped with a wealth of enviroment sensors similar to the Wolfbot mentioned in \ref{sec:wolfbot}. Replacing the wheels with omnidirectional ones would also give it more manueverablility or replacing the motors and using a real wheel encoder.

The platform is well suited for not very demanding visual SLAM methods using the available cameras, for such research the Jetbot mentioned in Section \ref{sec:jetbot} is better suited for it due to better access to the GPU. An upgrade to the newly released Raspberry Pi 4 would be well justified, because of greatly updated specifications, especially GPU, and support for OpenGL ES 3\footnotemark.

\footnotetext{https://opensource.com/article/19/6/raspberry-pi-4}


\end{document}
