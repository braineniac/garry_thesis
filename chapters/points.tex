\section{Electrical structure}\label{sec:elec_struct}
\begin{itemize}
\item Describe electrical layout and how everything is connected
\item Improvement:
 \begin{itemize}
  \item Battery voltage watcher->preferably beeping
  \item better connections and connectors for connecting new sensors and components
  \item 3d printed base with considerations for a battery holder and one switch to power off
  \item servos reacting because of self interference in wiring
 \end{itemize}
\end{itemize}

%%%%%%%%%%%%%%%%%%%%%%%%%%%%%%%%%%%%%%%%%%%%%%%%%%%%%%%%%%%%%%%%%%%%%%%%%%%%%%%%%%%%%%%%%%%%%%%%%
\section{SBC}\label{sec:sbc_lim}
\begin{itemize}
 \item compare Raspi to other SBCs
 \item compare to jetson nano, cuda toolkit
 \item RAM limitations->cross compile during development
 \item Ethernet and Wifi limitations
 \item storage durability->no onboard flash
 \item overheating when compiling
 \item USB power limitation
\end{itemize}

%%%%%%%%%%%%%%%%%%%%%%%%%%%%%%%%%%%%%%%%%%%%%%%%%%%%%%%%%%%%%%%%%%%%%%%%%%%%%%%%%%%%%%%%%%%%%%%%%
\section{Software}\label{software_lim}
\begin{itemize}
 \item compare ROS1 and ROS2
 \item driver availability for the raspberry and ROS support
 \item Raspbian vs Ubuntu 16.04 for the raspberry and the ubiquity ROS image
\end{itemize}

%%%%%%%%%%%%%%%%%%%%%%%%%%%%%%%%%%%%%%%%%%%%%%%%%%%%%%%%%%%%%%%%%%%%%%%%%%%%%%%%%%%%%%%%%%%%%%%%%
\section{IMU}\label{imu_lim}
\begin{itemize}
 \item Describe the drawbacks of the IMU and the drifting issue in general
\end{itemize}

%%%%%%%%%%%%%%%%%%%%%%%%%%%%%%%%%%%%%%%%%%%%%%%%%%%%%%%%%%%%%%%%%%%%%%%%%%%%%%%%%%%%%%%%%%%%%%%%%
\section{Motors}\label{motor_lim}
\begin{itemize}
 \item no feedback mechanism->rotary encoder
 \item cant go forward and turn at the same time
 \item generally loud
\end{itemize}

%%%%%%%%%%%%%%%%%%%%%%%%%%%%%%%%%%%%%%%%%%%%%%%%%%%%%%%%%%%%%%%%%%%%%%%%%%%%%%%%%%%%%%%%%%%%%%%%%
\section{Similar platforms}\label{sec:similar_platforms}
\begin{itemize}
 \item Find similar platforms and compare uses and price
\end{itemize}


\begin{itemize}
 \item Describe vaguely what a Kalman filter is and where it is used(with example applications)
 \item Explain why it is important in basic odometry and robotics
\end{itemize}

%%%%%%%%%%%%%%%%%%%%%%%%%%%%%%%%%%%%%%%%%%%%%%%%%%%%%%%%%%%%%%%%%%%%%%%%%%%%%%%%%%%%%%%%%%%%%%%%%
%%%%%%%%%%%%%%%%%%%%%%%%%%%%%%%%%%%%%%%%%%%%%%%%%%%%%%%%%%%%%%%%%%%%%%%%%%%%%%%%%%%%%%%%%%%%%%%%%
\section{Custom Kalman filter}\label{sec:custom_kalman}

\begin{itemize}
 \item Describe why we implemented a custom Kalman filter(insight to how it works I guess?)
 \item Maybe a better name for this section?
 \item add all the equations that describe the system from the code and add explanations with each step
 \item reference Kugi's script and how my system differs
\end{itemize}

%%%%%%%%%%%%%%%%%%%%%%%%%%%%%%%%%%%%%%%%%%%%%%%%%%%%%%%%%%%%%%%%%%%%%%%%%%%%%%%%%%%%%%%%%%%%%%%%%
\section{Extended Kalman Filter}\label{sec:extended_kalman}

\begin{itemize}
 \item Describe what it does and why it is better than my own Kalman filter
 \item Why do we also use this(reference later tests)
 \item Further reading in Kugi's script
\end{itemize}

%%%%%%%%%%%%%%%%%%%%%%%%%%%%%%%%%%%%%%%%%%%%%%%%%%%%%%%%%%%%%%%%%%%%%%%%%%%%%%%%%%%%%%%%%%%%%%%%%
\section{Simulation}\label{sec:sim}

\begin{itemize}
 \item Describe why the simulation was necessary and what purpose it serves.
 \item Describe what the real inputs look like(add ref) and why you chose the specific mathematical description(add equations)
 \item Hypotesize about the effects of covariances
\end{itemize}
\hrulefill
\begin{itemize}
 \item Before this input figure, describe the actual parameters in the equations and why you chose them
\end{itemize}

\begin{figure}[h]
 \begin{subfigure}[b]{0.45\textwidth}
  \input{plots/input_vel_sim.tex}\label{InputVelSim}
  \caption{Simulated input from joystick}
 \end{subfigure}
 \hfill
 \begin{subfigure}[b]{0.45\textwidth}
  \input{plots/input_accel_sim.tex}\label{InputAccelSim}
  \caption{Simulated input from accelerometer}
 \end{subfigure}
 \caption{Input simulation}
\end{figure}

\begin{itemize}
 \item Explain why the figures below look like this
 \item Describe the stdevs you chose and how they impacted the velocity(add ref for the actual comparison below)
 \item TODO: fix the $m/s^2 and 5*10^-2$ in figures
\end{itemize}

\begin{figure}[H]
 \flushleft
 \begin{subfigure}[c]{0.8\textwidth}
  \input{plots/robot_dist_sim.tex}\label{RobotDistSim}
  \caption{Kalman filter x state}
 \end{subfigure}
 \hfill
 \begin{subfigure}[c]{0.8\textwidth}
  \input{plots/robot_vel_sim.tex}\label{RobotVelSim}
  \caption{Kalman filter $\dot{x}$ state}
 \end{subfigure}
 \hfill
 \begin{subfigure}[c]{0.8\textwidth}
  \input{plots/robot_accel_sim.tex}\label{RobotAccelSim}
  \caption{Kalman filter $\ddot{x}$ state}
 \end{subfigure}
 \caption{Kalman filter states}
\end{figure}

\begin{itemize}
 \item Describe how changing the stdev ratio impacts the velocity(ref Kalman filter covariances)
 \item Add three velocity figures with >,<,= stdevs(a bit exaggerated ones)
 \item Explain why the variation of stdevs is important during different situations(standing still, moving)
\end{itemize}

%%%%%%%%%%%%%%%%%%%%%%%%%%%%%%%%%%%%%%%%%%%%%%%%%%%%%%%%%%%%%%%%%%%%%%%%%%%%%%%%%%%%%%%%%%%%%%%%%
\section{Driving in a straight line}\label{sec:straight_line}

\begin{itemize}
 \item Explain why this test is important->longer paths are made out of the sum of straight lines(garry cant turn and move forward, add ref to limitations)
\end{itemize}

\subsection{Simulation vs reality}\label{subsec:simvsreal}
\begin{itemize}
 \item Describe how the real input differs from the simulated one
 \item add two real input figures side by side
\end{itemize}
\hrulefill
\begin{itemize}
 \item Explain the differences between the real and simulated Kalman states(with static covar)
 \item add 3 figures like in the sim figure
\end{itemize}

\subsection{Covariance dependance}\label{subsec:covardep}
\begin{itemize}
 \item Explain why the adaptive covariance during the different phases(standing still, moving) smooths out the the $\dot{x}$ state during movement
 \item add two figures of the $\dot{x}$ state with static and adaptive covariance
\end{itemize}

%%%%%%%%%%%%%%%%%%%%%%%%%%%%%%%%%%%%%%%%%%%%%%%%%%%%%%%%%%%%%%%%%%%%%%%%%%%%%%%%%%%%%%%%%%%%%%%%%
\subsection{EKF vs custom Kalman filter comparison}\label{subsec:ekfvscustom}
\begin{itemize}
 \item add section for comparison between ekf and custom kalman filter
 \item add a $\dot{x}$ figure of the two side by side
 \item check out if they deal with variations in input velocity differently(+-20,+-50\%)
 \item add figures for differences
\end{itemize}

%%%%%%%%%%%%%%%%%%%%%%%%%%%%%%%%%%%%%%%%%%%%%%%%%%%%%%%%%%%%%%%%%%%%%%%%%%%%%%%%%%%%%%%%%%%%%%%%%
\subsection{Odometry dependance on acceleration}\label{subsec:odomdepaccel}
\begin{itemize}
 \item disable acceleration and compare that with the Kalman filter results
 \item add figure with comparison in distance
\end{itemize}

%%%%%%%%%%%%%%%%%%%%%%%%%%%%%%%%%%%%%%%%%%%%%%%%%%%%%%%%%%%%%%%%%%%%%%%%%%%%%%%%%%%%%%%%%%%%%%%%%
\section{Looping though an octagon}\label{sec:octagon}

\begin{itemize}
 \item Explain why we are doing this test->see drifts after each loop
 \item Describe testing parameters(setup of ekf package)
 \item Describe what happens in the figure below
 \item add figure
\end{itemize}


\begin{itemize}
 \item Describe the purpose of the platform->reference abstract?
 \item Describe platform in general parts terms(wheeled robot with imu, linux based SBC, microcontroller etc.)
\end{itemize}

%%%%%%%%%%%%%%%%%%%%%%%%%%%%%%%%%%%%%%%%%%%%%%%%%%%%%%%%%%%%%%%%%%%%%%%%%%%%%%%%%%%%%%%%%%%%%%%%%
\section{Hardware}
\pagestyle{scrheadings}

\begin{itemize}
 \item Explain the hardware choices and what we had to work with and what we chose to work with
 \item add pictures
\end{itemize}

This is just here so I remember the exact names, I would describe the part choices and use in separate paragraphs.
\begin{itemize}
 \item Raspberry Pi 3 Model B Rev 1.2
 \item Arduino Uno Rev3
 \item Arduino Motor Shield Rev3
 \item Raspberry Pi Night Vision Camera Module
 \item Logitech C905 Webcam
 \item SparkFun 9DoF Razor IMU M0
 \item Logitech Dual Action Gamepad
 \item 2x generic 5V DC motors
 \item 2x EMAX ES08MA II 12g Mini Metal Gear Analog Servo
\end{itemize}

%%%%%%%%%%%%%%%%%%%%%%%%%%%%%%%%%%%%%%%%%%%%%%%%%%%%%%%%%%%%%%%%%%%%%%%%%%%%%%%%%%%%%%%%%%%%%%%%%
\section{Software}
\pagestyle{scrheadings}

\begin{itemize}
 \item Describe the software stack: Linux+ROS+Arduino
 \item Why using linux+ROS makes sense instead of hardcoding everything, highlight ROS pros and contras
 \item Arduino platform and software uses here and in general-> comparison to full real time systems and microROS projects and how the comm bridge to ROS is implemented
\end{itemize}

%%%%%%%%%%%%%%%%%%%%%%%%%%%%%%%%%%%%%%%%%%%%%%%%%%%%%%%%%%%%%%%%%%%%%%%%%%%%%%%%%%%%%%%%%%%%%%%%%
\section{Expandability}
\pagestyle{scrheadings}

\begin{itemize}
 \item Describe the possible expansions of the platform for different applications
 \item Application ideas: 
 \begin{itemize}
 \item odometry based on vision->maybe moving the camera with the servos
 \item adding a distance measuring sensor and use it for mapping(lidar or ultrasonic)
 \item mapping pointclouds with the camera
 \item use both cameras for 360 degree simultaneous mapping
 \item other ideas?
 \end{itemize}
\end{itemize}


