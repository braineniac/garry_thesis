\documentclass[class=article, crop=false]{standalone}
\usepackage{import}

\begin{document}

\chapter{Motivation}\label{cha:motivation}

The rise in popularity of the maker culture, which democratises access to STEM and other tech-rich domains, has enabled widespread production of cheap components which can be utilised not just for hobby projects, but also for research purposes.

This project aims to use off the shelf components to build a rudimentary robot with basic odometrical capabilities while keeping its budget minimal, but extensible enough to be used in many applications like computer vision research, mapping and autonomous driving.

Through exploring the limits of such a construct, we can speculate what is reasonably possible to build with today's readily available maker products that can be used as a base for an extensible platform.

Our choice of components was dictated by what we already had available at the time of building and is not meant to be the most cost effective nor the best price per sensor accuracy solution. Almost every component of the platform is meant to be easily upgradeable to better suit the application or area of research.

\clearpage
\end{document}
