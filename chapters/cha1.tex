\documentclass[class=report, crop=false]{standalone}
\usepackage[subpreambles=true]{standalone}
\usepackage{import}

\begin{document}

\chapter{Introduction}\label{cha:introduction}

This project aims to use off the shelf components to build a rudimentary robot with basic odometrical capabilities with a minimal budget.

By constructing a robot with such limitations, we can speculate what is reasonably possible to build. Today's readily available maker products and off-the-shelf components are used as a base for an extensible platform for research and development.

Due to budget constraints we didn't use a wheel encoder commonly found in similar platforms. We explore the adequacy of using an inertial measurement sensor(IMU) and the signals from a joystick. We then refine the resulting state estimations with visual SLAM using our cameras.

Our choice of components was dictated by what we already had available at the time of building and is not intended to be the most cost effective nor the best price per sensor accuracy solution. Almost every component of the platform is  easily upgradeable to better suit the intended application.

Following this introduction we discuss platforms similar in components and price and in chapter 2, then in chapter 3 the hardware and software approach, in chapter 4 we present some experiments and evaluation, in chapter 5 we present our conclusion and in the end, chapter 6, an outlook for ways to tackle some of the issues and limitations.

\end{document}
