\chapter{Motivation}\label{cha:motivation}

What do you want to achieve (Introduction/Motivation)

Old abstract:\\
The rise in popularity of the maker culture, which democratises access to STEM and other tech-rich domains, has enabled widespread production of cheap components which can be utilised for research purposes.\\
This project aims to use off the shelf components to build a rudimentary robot with basic odometrical capabilities while keeping its budget minimal.\\
Through exploring the limits of such a construct, we can speculate what is reasonably possible to build with today's maker products that can
be used as a base for an extensible platform to conduct scientific and engineering research at an affordable price.

Old short summary:\\
The availability and low cost of maker products enable wide adoption and make it possible for most hobbyists and
researchers to build robots for their specific applications on the cheap.\\
For our robotic platform we chose readily available components, but kept expandability in mind. The current market
value of these hardware components was also considered to make it more reproducible at an affordable cost. Our choice
was dictated by what we already had available at the time of building and is not meant to be the most cost effective
nor the best price per accuracy solution. Almost every part of the platform is meant to be easily replaceable and can
be adapted for their intended application or area of research.\\
Basic odometry serves as a starting point for future projects that can expand its capabilities and improve its accuracy
by using multiple sensors, audio feedback or video processing.
