\documentclass[class=article, crop=false]{standalone}
\usepackage[subpreambles=true]{standalone}
\usepackage{import}
%\usepackage{booktabs}
%\usepackage{tikz}

%\usepackage[utf8]{inputenc}
\usepackage[subpreambles=true]{standalone}
\usepackage{import}
\usepackage{pgfplots}
\pgfplotsset{compat=newest}
\usepgfplotslibrary{groupplots}
\usepgfplotslibrary{dateplot}
\usepackage{caption}
\usepackage{subcaption}
\usepackage{graphicx}
\usepackage{amsmath}
\usepackage{amssymb}
\usepackage[parfill]{parskip}
\usepackage{float}

% \usepackage{pgfplots}
% \usetikzlibrary{pgfplots.groupplots}
% \pgfplotsset{compat=1.9,height=0.3\textheight,legend cell align=left,tick scale binop=\times}
% \pgfplotsset{grid style={loosely dotted,color=darkgray!30!gray,line width=0.6pt},tick style={black,thin}}
% \pgfplotsset{every axis plot/.append style={line width=0.8pt}}
%
% \usepgfplotslibrary{external}
% % Für die Verwendung von 'external' müssen die folgenden Anpassungen in Abhängigkeit der
% % LaTeX Distribution durchgeführt werden:
%
% % fuer Texlive: pdflatex.exe -shell-escape -synctex=1 -interaction=nonstopmode %.tex
% \tikzexternalize[shell escape=-shell-escape]   % fuer TeXLive
%
% % fuer MikTeX:  pdflatex.exe -enable-write18 -synctex=1 -interaction=nonstopmode %.tex
% %\tikzexternalize[shell escape=-enable-write18] % fuer MikTex
%
%
%
% \tikzsetexternalprefix{graphics/pgfplots/} % Ordner muss ev. zuerst haendisch erstellt werden


\begin{document}

\chapter{Introduction}\label{cha:introduction}
This project aims to use off the shelf components to build a rudimentary robot with basic odometrical capabilities while keeping its budget minimal. Through exploring the limits of such a construct, we can speculate what is reasonably possible to build with today's readily available maker products and off-the-shelf components that can be used as a base for an extensible platform for research and development.

Due to budget contraints we didn't use a wheel encoder commonly found in similar platforms and we explore the adequacy of using an inerial measurement sensor(IMU) and a fake wheel encoder implmented in software.

Our choice of components was dictated by what we already had available at the time of building and is not meant to be the most cost effective nor the best price per sensor accuracy solution. Almost every component of the platform is meant to be easily upgradeable to better suit the application or area of research.

In this thesis, we discuss platforms similar in price and components in chapter 2, then in chapter 3 the hardware and software used respectively, in chapter 4 we present some experiments and evaluation, in chapter 5 we present our conclusion and in end, chapter 6, an outlook for ways to tackle some of the issues and limitations.

\end{document}
