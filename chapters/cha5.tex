\documentclass[class=report, crop=false]{standalone}
\usepackage[subpreambles=true]{standalone}
\usepackage{import}

\begin{document}

\chapter{Conclusion}\label{cha:conclusion}

In this thesis we introduced a platform built out of remains of a toy robot and equipped it with components that make it a cost effective solution for research. Its current odometrical capabilities are based on a fake wheel encoder that is triggered by a Gamepad and the Razor IMU.

From the experiments we conducted in Sections \ref{sec:adapt} and \ref{sec:simplevsekf} we can conclude that odometry based on our fake wheel encoder is not well refined by our accelerometer since our robot is light and reaches its top velocity in a too short amount of time for including the acceleration in our model.

The experiment done in Section \ref{sec:circles} shows that the gyroscope drift causes the position to shift in a certain direction and should be continuously corrected by another sensor.


\end{document}
