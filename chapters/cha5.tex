\documentclass[class=report, crop=false]{standalone}
\usepackage[subpreambles=true]{standalone}
\usepackage{import}

\begin{document}

\chapter{Conclusion}\label{cha:conclusion}

In the thesis we introduced a platform built out of remains of a toy robot and equipped it with components that make it a cost effective solution for research.

We have implemented custom Kalman Filters for odometry using joystick signals as a control input and IMU sensors as an output.

From the conducted experiments we learned that small errors in the rotation accumulate easily and cause deformation of the overall trajectory.

We have also learned that correcting robot velocity with an accelerometer might lead to higher accuracy, but due to the short acceleration period and noise from the motors, a more expensive high frequency one should be used with some kind of dampening for the motor vibrations for best results.

\end{document}
