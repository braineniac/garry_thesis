\documentclass[class=report, crop=false]{standalone}
\usepackage[subpreambles=true]{standalone}
\usepackage{import}
%%\usepackage{booktabs}
%\usepackage{tikz}

%\usepackage[utf8]{inputenc}
\usepackage[subpreambles=true]{standalone}
\usepackage{import}
\usepackage{pgfplots}
\pgfplotsset{compat=newest}
\usepgfplotslibrary{groupplots}
\usepgfplotslibrary{dateplot}
\usepackage{caption}
%\usepackage{subcaption}
\usepackage{graphicx}
\usepackage{amsmath}
\usepackage{amssymb}
\usepackage[parfill]{parskip}
\usepackage{float}
\usepackage[bottom]{footmisc}

\setlength{\parindent}{2em}
\setlength{\parskip}{0.5em}
\usepackage{subcaption}
\usepackage{indentfirst}
\pgfplotsset{yticklabel style={text width=2em,align=right}}
\usepgfplotslibrary{external}
\usepackage{tikz}
\usepackage{shellesc}
\usetikzlibrary{external}
\tikzexternalize[shell escape=-enable-write18]
%\tikzexternalize[shell escape=-shell-escape]
\tikzset{external/system call={lualatex \tikzexternalcheckshellescape -halt-on-error -interaction=batchmode -jobname "\image" "\texsource"}}


% \usepackage{pgfplots}
% \usetikzlibrary{pgfplots.groupplots}
% \pgfplotsset{compat=1.9,height=0.3\textheight,legend cell align=left,tick scale binop=\times}
% \pgfplotsset{grid style={loosely dotted,color=darkgray!30!gray,line width=0.6pt},tick style={black,thin}}
% \pgfplotsset{every axis plot/.append style={line width=0.8pt}}
%
% \usepgfplotslibrary{external}
% % Für die Verwendung von 'external' müssen die folgenden Anpassungen in Abhängigkeit der
% % LaTeX Distribution durchgeführt werden:
%
% % fuer Texlive: pdflatex.exe -shell-escape -synctex=1 -interaction=nonstopmode %.tex
% \tikzexternalize[shell escape=-shell-escape]   % fuer TeXLive
%
% % fuer MikTeX:  pdflatex.exe -enable-write18 -synctex=1 -interaction=nonstopmode %.tex
% %\tikzexternalize[shell escape=-enable-write18] % fuer MikTex
%
%
%
\tikzsetexternalprefix{graphics/} % Ordner muss ev. zuerst haendisch erstellt werden


\begin{document}

\chapter{Einführung}\label{cha:einfuhrung}

Dieses Projekt zielt darauf ab, aus Standardkomponenten einen rudimentären Roboter mit grundlegenden odometrischen Fähigkeiten mit minimalem Budget zu bauen.

Durch die Herstellung eines Roboters mit solchen Einschränkungen kann abgeschätzt werden, was auf vernünftige Weise baubar ist.
Als Basis für eine erweiterbare Plattform für Forschung und Entwicklung dienen heutzutage leicht verfügbare Evaluierungsboards und Standardkomponenten.

Aus Kostengründen wird auf Drehgeber verzichtet, wie er bei ähnlichen Plattformen üblich ist; die Adäquatheit wird von Trägheitsmesssensoren und Joysticksignalens untersucht. Anschließend wird die resultierende Zustandschätzung mit dem visuellen SLAM unserer Kameras verfeinert.

Unsere Komponentenwahl wird dadurch bestimmt, welche Teile uns zum Zeitpunkt der Konstruktion zur Verfügung standen. Weder steht im Vordergrund, die kostengünstigste Variante zu produzieren, noch ist das beste Preis-Leistungs-Verhältnis im Bereich der Sensorikgenauigkeit das Ziel. Fast alle Komponenten der Plattform sind leicht erweiterbar, um für die gezielte Anwendung besser anpassbar zu sein.

Im Anschluss an die Einleitung werden in Kapitel 2 dieser Arbeit sämtliche Plattformen mit einer ähnlichen Komponentenwahl und ähnlichen Kosten diskutiert, in Kapitel 3 wird der Hard-und Software-Ansatz besprochen. In Kapitel 4 werden etliche Experimente und Auswertungen präsentiert. Kapitel 5 fasst die Ergebnisse dieser Arbeit zusammen, die Grundlage für das abschließende Kapitel 6 sind, in dem ein Ausblick darauf gegeben wird, wie die Probleme und Einschränkungen der Plattform gelöst werden könnten.

\end{document}
