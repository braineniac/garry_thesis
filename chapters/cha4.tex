\documentclass[class=report, crop=false]{standalone}
%\usepackage[subpreambles=true]{standalone}
\usepackage{import}
%\usepackage{booktabs}
%\usepackage{tikz}

%\usepackage[utf8]{inputenc}
\usepackage[subpreambles=true]{standalone}
\usepackage{import}
\usepackage{pgfplots}
\pgfplotsset{compat=newest}
\usepgfplotslibrary{groupplots}
\usepgfplotslibrary{dateplot}
\usepackage{caption}
\usepackage{subcaption}
\usepackage{graphicx}
\usepackage{amsmath}
\usepackage{amssymb}
\usepackage[parfill]{parskip}
\usepackage{float}

% \usepackage{pgfplots}
% \usetikzlibrary{pgfplots.groupplots}
% \pgfplotsset{compat=1.9,height=0.3\textheight,legend cell align=left,tick scale binop=\times}
% \pgfplotsset{grid style={loosely dotted,color=darkgray!30!gray,line width=0.6pt},tick style={black,thin}}
% \pgfplotsset{every axis plot/.append style={line width=0.8pt}}
%
% \usepgfplotslibrary{external}
% % Für die Verwendung von 'external' müssen die folgenden Anpassungen in Abhängigkeit der
% % LaTeX Distribution durchgeführt werden:
%
% % fuer Texlive: pdflatex.exe -shell-escape -synctex=1 -interaction=nonstopmode %.tex
% \tikzexternalize[shell escape=-shell-escape]   % fuer TeXLive
%
% % fuer MikTeX:  pdflatex.exe -enable-write18 -synctex=1 -interaction=nonstopmode %.tex
% %\tikzexternalize[shell escape=-enable-write18] % fuer MikTex
%
%
%
% \tikzsetexternalprefix{graphics/pgfplots/} % Ordner muss ev. zuerst haendisch erstellt werden


\begin{document}

\chapter{Experiments}

In our experiments with the platform we show the tuning process of the parameters described in Section \ref{sec:kalman} and see how our adaptive Kalman Filter performs compared to the de facto standard non-linear Kalman Filter, the Extended Kalman Filter.

During the parameter tuning experiments, we first pick a reference experiment on which we base the tuning of our parameter, then we test it in different but also similar scenarios to see how well it estimates the real state of the robot.

After choosing a parameter value we run the same experiment with the Extended Kalman Filter(EKF) to see how switching the Kalman Filter effects the states with the same parameters. We inserted a node between the sensor nodes and EKF to scale our input vector $\textbf{u}_k$ with $\alpha$ and $\beta$, and set the covariances of our sensors. The node ran with a rate of 10Hz to properly use the sensor data.

All of the experiments were performed by recording the sensor data on the robot while performing the experiment. Then we transformed the sensor data to the same reference point and ran the Kalman Filters offline. This process was automated to ensure paramater consistency during each run, but could also be easily run online on the robot.

\import{sections/}{exp_alpha}

\import{sections/}{exp_beta}

\import{sections/}{exp_floor}

\import{sections/}{exp_octagon}

\end{document}
